\documentclass[11pt]{article}
\usepackage[french]{babel}
\usepackage[utf8]{inputenc}
\usepackage[T1]{fontenc}
\usepackage{times}
\usepackage{fullpage}
\usepackage{listings}
\usepackage{url}

\lstset{language=C}
\lstset{basicstyle=\footnotesize\ttfamily,
  keywordstyle=\bfseries,
  keywordstyle=[2]\it}
\lstset{showstringspaces=false}


\title{Première séance machine\\Notions de C, compilation, débuggage}
\author{Ensimag 1A Apprentissage, Logiciel de Base}
\date{mai 2011}

\begin{document}
\maketitle

Tous les programmes et morceaux de code sont disponibles dans le
répertoire \texttt{c/1b-gdb-valgrind/} de l'archive Git.

\section{Premiers pas en C}

On commence avec un programme simple, \texttt{hello.c} :

\lstinputlisting{hello.c}

(la fonction \texttt{puts} affiche une chaîne à l'écran, puis passe à
la ligne. La fonction \texttt{printf} affiche la chaîne donnée en
argument, dans laquelle on remplace les \texttt{"\%d"} par les
arguments suivant la chaîne)

% Remarque : on utilise puts, parce que gcc fait une optimisation
% lorsqu'il voit un printf avec une chaine de formatage sans % et
% finissant par \n, il le remplace par le puts équivalent. Du coup,
% les étudiants ne vont plus rien comprendre ...

Compilez-le, puis exécutez-le :

\begin{verbatim}
telesun> gcc hello.c -o hello
telesun> ./hello
[...]
\end{verbatim}

\section{Exploration de la chaîne de compilation}

\subsection{Compilation}

Il s'est en fait passé beaucoup de choses dans les deux commandes
précédentes. La première étape est la compilation proprement dite, on
peut arrêter \texttt{gcc} après cette étape avec l'option
\texttt{-S} :

\begin{verbatim}
telesun> gcc -S hello.c
\end{verbatim}

Ceci va générer un fichier \texttt{hello.s}. Ouvrez-le dans un éditeur
de texte. Vous avez devant vous un programme en langage d'assemblage.
On trouve dedans :

\begin{itemize}
\item Des définitions de données (les chaînes qui apparaissent dans le
  programme)
\item Des instructions. En particulier, on reconnait les appels à la
  fonction \texttt{puts}, qui sont sous la forme \texttt{call puts},
  entourées d'incantations magiques.
\item Des étiquettes, par exemple, \verb|main:|.
\end{itemize}

A ce stade, le programme ne dit rien sur l'implémentation des
fonctions \texttt{puts}, \texttt{printf} et \texttt{scanf}. On se
contente de supposer qu'elles existent (on cherchera leur
implémentation plus tard, c'est l'édition de liens).

\subsection{Assemblage}

Le langage d'assemblage décrit précisément chaque instruction à
exécuter, mais elles ne sont pas dans un format directement exécutable
par le processeur. L'étape suivante est d'assembler les instructions,
on peut le faire avec la commande :

\begin{verbatim}
telesun> gcc -c hello.s
\end{verbatim}

Ce qui va créer un fichier \texttt{hello.o}. Essayez de l'ouvrir dans
un éditeur de texte pour vous convaincre que c'est du binaire
illisible. On reconnaît quand même quelques chaînes de caractères. On
peut afficher le fichier de manière un peu raisonnable avec des
commandes comme :

\begin{verbatim}
telesun> hexdump hello.o
[...]
telesun> hexdump -c hello.o
[...]
\end{verbatim}

Mais des outils spécialisés peuvent aussi extraire quelques données
intéressantes du fichier. Par exemple, la commande \texttt{nm} liste
les symboles définis ou utilisés dans le fichier (l'équivalent des
étiquettes, mais après assemblage) :

\begin{verbatim}
$ nm hello.o
00000000 T main
         U printf
         U puts
         U scanf
\end{verbatim}
%$

On voit que le symbole \texttt{main} est défini (<<~T~>> veut dire
<<~Text~>>, on verra plus tard pourquoi). En revanche, les symboles
\texttt{printf}, \texttt{puts} et \texttt{scanf} sont toujours
indéfinis (<<~U~>> pour <<~Undefined~>>).

La traduction du langage d'assemblage vers le binaire garde la
quasi-totalité des informations du fichiers assembleur, mais change de
représentation. On peut retrouver le texte assembleur avec un
désassembleur, par exemple :

\begin{verbatim}
$ objdump -d hello.o
\end{verbatim}
%$

(objdump n'est pas installé par défaut sous Mac OS X. Si vous ne
pouvez pas faire cette manipulation, ce n'est pas très grave ...)

\subsection{Édition de lien}

L'étape suivante est surtout intéressante lorsqu'on fait de la
compilation séparée, mais il se passe aussi des choses avec un seul
fichier : on <<~prépare~>> le fichier, pour qu'il soit prêt à être
exécuté :

\begin{verbatim}
telesun> gcc hello.o -o hello
\end{verbatim}

Le fichier généré est un peu plus gros. La commande \texttt{nm} donne
beaucoup plus de symboles. Par exemple, il y a maintenant un symbole
\texttt{\_start}, qui correspond au point d'entrée de l'exécution
(c'est lui qui appellera la fonction \texttt{main}, au tout début de
l'exécution). L'éditeur de lien a également repéré où se trouveront
les fonctions \texttt{printf}, \texttt{puts}, \texttt{scanf} (et il
aurait donné une erreur si il ne les avait pas trouvés). On peut voir
ça avec la commande \texttt{ldd} (ou \texttt{otool -L} sous Mac) :

\begin{verbatim}
telesun> ldd hello
        libc.so.6 => /lib/tls/libc.so.6 (0x40018000)
        /lib/ld-linux.so.2 (0x40000000)
\end{verbatim}

On n'utilise que des fonctions de la bibliothèque C standard, donc il
y a une référence à \texttt{libc.so} (ne vous occupez pas du
\texttt{ld-linux.so}). Mais sur un exécutable de la vraie vie, c'est un
peu plus long. Par exemple, pour Emacs :

\begin{verbatim}
telesun> ldd /usr/bin/emacs
[...]
\end{verbatim}

Il y a plus de monde !

\section{Makefile}

Lorsqu'il n'y a qu'un seul fichier à compiler, on a vu plus haut
qu'une commande suffisait à tout faire (\texttt{gcc fichier.c -o
  executable}), mais les choses se compliquent sur des gros projets :
si le projet met 1 heure à compiler, on ne veut pas {\em tout}
recompiler à chaque fois, on veut un outil qui recompile seulement ce
qui est nécessaire.

L'archive disponible sur le site du cours contient un répertoire {\tt
  compilation/}.

Le répertoire contient 3 fichiers sources \texttt{.c}, et deux fichiers
d'en-têtes \texttt{.h}. Un Makefile très simple est fourni, pour
l'utiliser, il suffit de taper \texttt{make} dans le répertoire du
projet. \texttt{make} va lire le fichier \texttt{Makefile}, regarder
les dates de dernière modifications des fichiers sur le disque, et en
déduire ce qu'il doit faire. Tapez \texttt{make} deux fois. La
première fois recompile tout, la seconde est presque instantance.

Maintenant, modifiez, par exemple, le fichier \texttt{bonjour.c} pour
changer le message affiché à l'utilisateur. Sauvez ce fichier, puis
recompilez avec \texttt{make} : le fichier est recompilé, et l'édition
de lien est refaite. Essayez la même chose en modifiant
\texttt{bonjour.h} : tous les fichiers qui en dépendent sont
recompilés.

\section{Utilisation de GDB pour déboguer des programmes}

Déboguer des programmes en C est bien plus difficile qu'en Ada ou en
Java (en langage d'assemblage, ça sera encore pire ...). D'une part
parce qu'il y a plus d'occasion de faire des erreurs,
mais aussi parce qu'un programme écrit en C qui plante ne donne pas le
numéro de ligne ou le nom de la fonction dans lequel s'est produit le
problème (typiquement, le programme dit juste <<~Segmentation Fault~>>).

Un mauvais programmeur ajouterait des \lstinline{printf} un peu
partout dans son code pour essayer de comprendre son exécution, mais
cette technique est très inefficace. Heureusement, il existe des
outils pour éviter ce calvaire : les débogueurs. Nous allons apprendre
à nous servir de {\tt gdb} : <<~GNU DeBugger~>>.

Ouvrez le fichier {\tt gdb/gdb-tutorial.c} dans votre éditeur de texte
favori, et suivez les instructions pas à pas.

\section{Valgrind: Un autre outil bien pratique pour l'aide au débogage}

Valgrind est un debogueur mémoire. Il permet d'exécuter un programme
en vérifiant les accès mémoires, et permet donc d'identifier des
erreurs de programmations qui seraient passées inaperçues « par
chance », ou d'expliquer certains comportements étranges d'un
programme. Une introduction courte est disponible sur EnsiWiki :
\url{https://ensiwiki.ensimag.fr/index.php/Valgrind}

On vous fournit aussi un module implantant une table de hachage et un
programme de test (très incomplet !). Compilez ce programme et
exécutez le : il plante avec une segmentation fault.

Vous allez devoir étudier le code du module table de hachage pour
comprendre comment il fonctionne et localiser les erreurs. Il s'agit
d'erreurs en liaison avec la gestion mémoire : vous avez donc tout
intérêt à utiliser le logiciel valgrind pour vous aider !

Indication : il y a 3 erreurs volontaires (et un nombre indéfini de
bugs non voulus ;) ) dans le module table de hachage.

\section{Compilation et exécution des exemples vus en TD}

Si vous ne l'avez pas déjà fait, téléchargez les exemples vus en TD
puis compilez et exécutez les programmes.

Si le temps le permet, essayez d'exécuter pas-à-pas les programmes
dans Gdb. Vous pouvez aussi jouer avec le fichier {\tt gdb/bug.c} dans
l'archive, et chercher à corriger le (ou plutôt les) bugs qu'il
contient.


% \section{Programmes plus évolués en C}

% \subsection{Entrées-sorties}

% En utilisant les fonctions {\tt printf} et {\tt scanf} (faire par
% exemple {\tt man 3 printf} pour une documentation), faites un
% programme qui dit bonjour à l'utilisateur, lui demande son âge et son
% prénom, et affiche <<Bonjour, XXX, vous avez YYY ans>>.

% Note: dans la vraie vie, l'utilisation de {\tt scanf} est en fait
% déconseillée. Pensez à ce qui se passe si l'utilisateur entre un
% prénom de 10000 caractères dans le programme que vous venez d'écrire.

% \subsection{Solutions du TD}

% Reprenez la feuille de TD de langage C, et implémentez sur machine les
% solutions des exercices. Autant que possible, faites vraiment
% l'exercice sur machine, sans regarder le corrigé.

\end{document}


%%% Local Variables: 
%%% mode: latex
%%% TeX-master: t
%%% ispell-local-dictionary: "francais"
%%% End: 
